% Template for paper format at JPA
\documentclass[11pt,a4paper,xelatex,ja=standard]{bxjsarticle}
\geometry{left = 5.4cm, right = 5.4cm, top = 4.0cm, bottom = 3.5cm}
\renewcommand{\figurename}{Figure}
\renewcommand{\tablename}{Table}
\usepackage{lmodern}
\usepackage{amssymb,amsmath}
\usepackage{ifxetex,ifluatex}
\ifnum 0\ifxetex 1\fi\ifluatex 1\fi=0 % if pdftex
  \usepackage[T1]{fontenc}
  \usepackage[utf8]{inputenc}
  \usepackage{textcomp} % provide euro and other symbols
\else % if luatex or xetex
  \usepackage{unicode-math}
  \defaultfontfeatures{Scale=MatchLowercase}
  \defaultfontfeatures[\rmfamily]{Ligatures=TeX,Scale=1}
\fi
% Use upquote if available, for straight quotes in verbatim environments
\IfFileExists{upquote.sty}{\usepackage{upquote}}{}
\IfFileExists{microtype.sty}{% use microtype if available
  \usepackage[]{microtype}
  \UseMicrotypeSet[protrusion]{basicmath} % disable protrusion for tt fonts
}{}
\makeatletter
\@ifundefined{KOMAClassName}{% if non-KOMA class
  \IfFileExists{parskip.sty}{%
    \usepackage{parskip}
  }{% else
    \setlength{\parindent}{0pt}
    \setlength{\parskip}{6pt plus 2pt minus 1pt}}
}{% if KOMA class
  \KOMAoptions{parskip=half}}
\makeatother
\usepackage{xcolor}
\IfFileExists{xurl.sty}{\usepackage{xurl}}{} % add URL line breaks if available
\IfFileExists{bookmark.sty}{\usepackage{bookmark}}{\usepackage{hyperref}}
\hypersetup{
  pdftitle={RMarkdownで『心理学研究』の論文は書けるのか?},
  hidelinks,
  pdfcreator={LaTeX via pandoc}}
\urlstyle{same} % disable monospaced font for URLs
\usepackage{color}
\usepackage{fancyvrb}
\newcommand{\VerbBar}{|}
\newcommand{\VERB}{\Verb[commandchars=\\\{\}]}
\DefineVerbatimEnvironment{Highlighting}{Verbatim}{commandchars=\\\{\}}
% Add ',fontsize=\small' for more characters per line
\usepackage{framed}
\definecolor{shadecolor}{RGB}{248,248,248}
\newenvironment{Shaded}{\begin{snugshade}}{\end{snugshade}}
\newcommand{\AlertTok}[1]{\textcolor[rgb]{0.94,0.16,0.16}{#1}}
\newcommand{\AnnotationTok}[1]{\textcolor[rgb]{0.56,0.35,0.01}{\textbf{\textit{#1}}}}
\newcommand{\AttributeTok}[1]{\textcolor[rgb]{0.77,0.63,0.00}{#1}}
\newcommand{\BaseNTok}[1]{\textcolor[rgb]{0.00,0.00,0.81}{#1}}
\newcommand{\BuiltInTok}[1]{#1}
\newcommand{\CharTok}[1]{\textcolor[rgb]{0.31,0.60,0.02}{#1}}
\newcommand{\CommentTok}[1]{\textcolor[rgb]{0.56,0.35,0.01}{\textit{#1}}}
\newcommand{\CommentVarTok}[1]{\textcolor[rgb]{0.56,0.35,0.01}{\textbf{\textit{#1}}}}
\newcommand{\ConstantTok}[1]{\textcolor[rgb]{0.00,0.00,0.00}{#1}}
\newcommand{\ControlFlowTok}[1]{\textcolor[rgb]{0.13,0.29,0.53}{\textbf{#1}}}
\newcommand{\DataTypeTok}[1]{\textcolor[rgb]{0.13,0.29,0.53}{#1}}
\newcommand{\DecValTok}[1]{\textcolor[rgb]{0.00,0.00,0.81}{#1}}
\newcommand{\DocumentationTok}[1]{\textcolor[rgb]{0.56,0.35,0.01}{\textbf{\textit{#1}}}}
\newcommand{\ErrorTok}[1]{\textcolor[rgb]{0.64,0.00,0.00}{\textbf{#1}}}
\newcommand{\ExtensionTok}[1]{#1}
\newcommand{\FloatTok}[1]{\textcolor[rgb]{0.00,0.00,0.81}{#1}}
\newcommand{\FunctionTok}[1]{\textcolor[rgb]{0.00,0.00,0.00}{#1}}
\newcommand{\ImportTok}[1]{#1}
\newcommand{\InformationTok}[1]{\textcolor[rgb]{0.56,0.35,0.01}{\textbf{\textit{#1}}}}
\newcommand{\KeywordTok}[1]{\textcolor[rgb]{0.13,0.29,0.53}{\textbf{#1}}}
\newcommand{\NormalTok}[1]{#1}
\newcommand{\OperatorTok}[1]{\textcolor[rgb]{0.81,0.36,0.00}{\textbf{#1}}}
\newcommand{\OtherTok}[1]{\textcolor[rgb]{0.56,0.35,0.01}{#1}}
\newcommand{\PreprocessorTok}[1]{\textcolor[rgb]{0.56,0.35,0.01}{\textit{#1}}}
\newcommand{\RegionMarkerTok}[1]{#1}
\newcommand{\SpecialCharTok}[1]{\textcolor[rgb]{0.00,0.00,0.00}{#1}}
\newcommand{\SpecialStringTok}[1]{\textcolor[rgb]{0.31,0.60,0.02}{#1}}
\newcommand{\StringTok}[1]{\textcolor[rgb]{0.31,0.60,0.02}{#1}}
\newcommand{\VariableTok}[1]{\textcolor[rgb]{0.00,0.00,0.00}{#1}}
\newcommand{\VerbatimStringTok}[1]{\textcolor[rgb]{0.31,0.60,0.02}{#1}}
\newcommand{\WarningTok}[1]{\textcolor[rgb]{0.56,0.35,0.01}{\textbf{\textit{#1}}}}
\setlength{\emergencystretch}{3em} % prevent overfull lines
\providecommand{\tightlist}{%
  \setlength{\itemsep}{0pt}\setlength{\parskip}{0pt}}
\setcounter{secnumdepth}{-\maxdimen} % remove section numbering
\usepackage{booktabs}
\usepackage{longtable}
\usepackage{array}
\usepackage{multirow}
\usepackage{wrapfig}
\usepackage{float}
\usepackage{colortbl}
\usepackage{pdflscape}
\usepackage{tabu}
\usepackage{threeparttable}
\usepackage{threeparttablex}
\usepackage[normalem]{ulem}
\usepackage{makecell}
\usepackage{xcolor}
\newlength{\cslhangindent}
\setlength{\cslhangindent}{1.5em}
\newenvironment{cslreferences}%
  {\setlength{\parindent}{0pt}%
  \everypar{\setlength{\hangindent}{\cslhangindent}}\ignorespaces}%
  {\par}
\title{RMarkdownで『心理学研究』の論文は書けるのか?}
\usepackage{etoolbox}
\makeatletter
\providecommand{\subtitle}[1]{% add subtitle to \maketitle
  \apptocmd{\@title}{\par {\large #1 \par}}{}{}
}
\makeatother
\subtitle{Can we write a paper of the Japanses Journal of psychology
with RMarkdown?}
\author{}
\date{}
\usepackage{marginnote}
\usepackage{titlesec}
\titleformat{\section}[hang]{\large\filcenter\bfseries}{\thesection}{1zw}{}

\begin{document}
\pagestyle{empty}
\maketitle
\pagestyle{plain}
\setcounter{page}{1}

\hypertarget{abstract}{%
\section{Abstract}\label{abstract}}

Can we write a paper of the Japanses Journal of psychology with
RMarkdown? To solve this mystery we headed deep into the Amazon. At the
end of our long journey we found some great documents about RMarkdown
and finally said, ``Yes, we can''. \ldots{} Well, we'll write a abstract
like this.

\textbf{Key words}: RMarkdown, Reproducibility, The Japanese Journal of
Psychology

\clearpage

 はじめに,ここから文章を書き始めます。以降が論文の本文になります。どんどん書いていきましょう!

\hypertarget{ux5fc3ux7406ux5b66ux306bux304aux3051ux308bux518dux73feux53efux80fdux6027}{%
\subsection{心理学における再現可能性}\label{ux5fc3ux7406ux5b66ux306bux304aux3051ux308bux518dux73feux53efux80fdux6027}}

 心理学の再現可能性はとってもまずい状況なのですが,それはちょっとおいておいて,文献の引用の仕方を説明します。まず,Kunisato
et al. (2012)
のように,すると,bibファイル内のKunisatoの2012年の論文が引用されます。そして,次のように,{[}{]}でくくると文末の引用スタイルになります(国里
et al. 2019)。また,文末に複数引用する場合は,こういう感じにします(国里
et al. 2019; Machino et al.
2014)。以下に詳しく書いているのでご確認ください。

 心理学の再現可能性はとってもまずい状況なのですが,それはちょっとおいておいて,文献の引用の仕方を説明します。まず,Kunisato
et al. (2012)
のように,すると,bibファイル内のKunisatoの2012年の論文が引用されます。そして,次のように,{[}{]}でくくると文末の引用スタイルになります(国里
et al. 2019)。また,文末に複数引用する場合は,こういう感じにします(国里
et al. 2019; Machino et al.
2014)。以下に詳しく書いているのでご確認ください。  

 \href{mailto:そしてまさかの@Lewinからの@Ekman1969ですよ}{\nolinkurl{そしてまさかの@Lewinからの@Ekman1969ですよ}}。(坂野
et al. 1994, @lerner2015handbook)  

\hypertarget{ux5f15ux7528ux6587ux732e}{%
\section{引用文献}\label{ux5f15ux7528ux6587ux732e}}

\hypertarget{refs}{}
\begin{cslreferences}
\leavevmode\hypertarget{ref-Kunisato2012}{}%
Kunisato, Yoshihiko, Yasumasa Okamoto, Kazutaka Ueda, Keiichi Onoda, Go
Okada, Shinpei Yoshimura, Shin-Ichi Suzuki, et al. 2012. ``Effects of
Depression on Reward-Based Decision Making and Variability of Action in
Probabilistic Learning.'' \emph{Journal of Behavior Therapy and
Experimental Psychiatry} 43 (4): 1088--94.

\leavevmode\hypertarget{ref-lerner2015handbook}{}%
Lerner, Richard, M, S. Lynn Liben, and Ulrich Mueller. 2015.
\emph{Handbook of Child Psychology and Developmental Science, Cognitive
Processes}. John Wiley \& Sons.

\leavevmode\hypertarget{ref-Machino2014}{}%
Machino, Akihiko, Yoshihiko Kunisato, Tomoya Matsumoto, Shinpei
Yoshimura, Kazutaka Ueda, Yosuke Yamawaki, Go Okada, Yasumasa Okamoto,
and Shigeto Yamawaki. 2014. ``Possible Involvement of Rumination in Gray
Matter Abnormalities in Persistent Symptoms of Major Depression: An
Exploratory Magnetic Resonance Imaging Voxel-Based Morphometry Study.''
\emph{Journal of Affective Disorders} 168 (October): 229--35.

\leavevmode\hypertarget{ref-kunisato2019}{}%
国里愛彦, 片平健太郎, 沖村宰, and 山下祐一. 2019.
``うつに対する計算論的アプローチ:―強化学習モデルの観点から―.''
\emph{心理学評論} 62 (1): 88--103.
\url{https://doi.org/10.24602/sjpr.62.1_88}.

\leavevmode\hypertarget{ref-ux5742ux91ceux96c4ux4e8c1994}{}%
坂野雄二, 福井知美, 熊野宏昭, 堀江はるみ, 川原健資, 山本晴義, 野村忍,
and 末松弘行. 1994.
``新しい気分調査票の開発とその信頼性・妥当性の検討.'' \emph{心身医学} 34
(8): 629--36.
\end{cslreferences}

\begin{Shaded}
\begin{Highlighting}[]
\KeywordTok{jpa\_cite}\NormalTok{(}\StringTok{"testRMD2.Rmd"}\NormalTok{,}\StringTok{"sample.bib"}\NormalTok{)}
\end{Highlighting}
\end{Shaded}

\begin{verbatim}
[1] "Kunisato,Yoshihiko, Okamoto,Y., Ueda,K., Onoda,K., Okada,G., Yoshimura,S....Samejima,Kusuo (2012). {\\emph Effects of depression on reward-based decision making and variability of action in probabilistic learning}, Journal of behavior therapy and experimental psychiatry, 43(4), 1088--1094."                                          
[2] "Machino,A., Kunisato,Y., Matsumoto,T., Yoshimura,S., Ueda,K., Yamawaki,Y....Yamawaki,S. (2014). {\\emph Possible involvement of rumination in gray matter abnormalities in persistent symptoms of major depression: an exploratory magnetic resonance imaging voxel-based morphometry study}, Journal of affective disorders, 168, 229--235."
[3] "国里 愛彦・片平健太郎・沖村宰・山下祐一 (2019). うつに対する計算論的アプローチ:―強化学習モデルの観点から― 心理学評論, {\\emph 62}(1), 88-103."                                                                                                                                                                                             
[4] "Lewin Kurt (1951). Field theory in social science: Selected theoretical papers , New York : Harper \\& Brothers(レヴィン・K (1951). 社会科学における場の理論 誠信書房)"                                                                                                                                                                      
[5] "坂野 雄二・福井知美・熊野宏昭・堀江はるみ・川原健資・山本晴義…末松弘行 (1994). 新しい気分調査票の開発とその信頼性・妥当性の検討 心身医学, {\\emph 34}(8), 629--636."                                                                                                                                                                        
[6] "Lerner,R.M., Liben,S.L. & Mueller,U. (2015). Handbook of child psychology and developmental science, cognitive processes , NA : John Wiley \\& Sons"                                                                                                                                                                                         
\end{verbatim}

\begin{Shaded}
\begin{Highlighting}[]
\CommentTok{\#paste(readLines("tmpbib.tex"), collapse = "\textbackslash{}n") \%\textgreater{}\% cat()}
\end{Highlighting}
\end{Shaded}


\end{document}
